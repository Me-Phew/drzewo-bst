\newpage
\section{Wnioski}

W ramach przeprowadzonego projektu opracowano aplikację do zarządzania drzewem binarnym wyszukiwania (BST), której głównym celem było umożliwienie użytkownikowi wykonywania podstawowych operacji na drzewie, takich jak dodawanie, usuwanie elementów oraz wyświetlanie struktury drzewa. Po zakończeniu implementacji i przeprowadzeniu testów można sformułować następujące wnioski:

\subsection{Cel projektu i jego realizacja}

Celem projektu było zaprojektowanie i implementacja algorytmu, który umożliwia efektywne zarządzanie drzewem binarnym wyszukiwania oraz udostępnienie intuicyjnego interfejsu użytkownika w postaci menu tekstowego. Projekt został zrealizowany zgodnie z założeniami, a aplikacja umożliwia wykonanie podstawowych operacji na drzewie BST w sposób wydajny i bezbłędny. Wszystkie funkcje aplikacji, takie jak dodawanie, usuwanie oraz przeglądanie drzewa, działają zgodnie z oczekiwaniami.

\subsection{Możliwości rozwoju i optymalizacji}

Choć aplikacja spełnia swoje zadanie, istnieje kilka obszarów, które mogą zostać zoptymalizowane w przyszłości:

\begin{itemize}
	\item \textbf{Zbalansowanie drzewa:} Algorytm nie zapewnia automatycznego balansowania drzewa, co może prowadzić do pogorszenia wydajności operacji, szczególnie w przypadku wstawiania elementów w sposób uporządkowany. Zastosowanie algorytmów balansujących, takich jak drzewa AVL czy czerwono-czarne, mogłoby poprawić czas wykonywania operacji.
	\item \textbf{Interfejs użytkownika:} Aplikacja wykorzystuje jedynie interfejs tekstowy, co ogranicza jej funkcjonalność i wygodę użytkowania. Można rozważyć rozbudowę aplikacji o graficzny interfejs użytkownika (GUI), co poprawiłoby interaktywność programu.
	\item \textbf{Zarządzanie pamięcią:} W przyszłości warto rozważyć implementację inteligentnego zarządzania pamięcią, szczególnie w przypadku dużych drzew. Zastosowanie wskaźników smart pointers mogłoby pomóc w automatycznym zarządzaniu pamięcią, co zmniejszyłoby ryzyko wycieków pamięci.
\end{itemize}

\subsection{Podsumowanie}

Projekt umożliwił dogłębne zrozumienie i implementację algorytmu dla drzewa binarnego wyszukiwania, a także pozwolił na praktyczne zapoznanie się z zagadnieniami związanymi z operacjami na drzewach. Choć aplikacja działa zgodnie z wymaganiami, przyszła wersja projektu może zostać rozbudowana o dodatkowe funkcjonalności, optymalizacje i udoskonalenia, które umożliwią jej jeszcze lepszą skalowalność i wygodę użytkowania.

Przeprowadzona implementacja pokazuje, że algorytm wyszukiwania w drzewie binarnym jest jednym z fundamentalnych narzędzi w informatyce, które znajduje szerokie zastosowanie w różnych dziedzinach, takich jak bazy danych, kompilatory czy systemy plików. Projekt stanowi solidną podstawę do dalszych badań nad algorytmami strukturalnymi i ich zastosowaniem w praktyce.

\nocite{GitHubProject, GitHubProjectTemplate}
