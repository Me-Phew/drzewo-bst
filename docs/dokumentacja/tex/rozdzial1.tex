\newpage
\section{Ogólne określenie wymagań}

Celem projektu jest stworzenie aplikacji w języku C++ umożliwiającej obsługę drzewa wyszukiwań binarnych (BST - Binary Search Tree) oraz umożliwiającej jego zarządzanie i wizualizację. Projekt realizowany jest z myślą o zdobyciu praktycznego doświadczenia z systemem kontroli wersji GitHub, przy pracy w grupie dwuosobowej, z zastosowaniem narzędzi do zarządzania kodem i dokumentacją.

\subsection{Zakładany efekt końcowy}

Efektem końcowym pracy jest aplikacja spełniająca wymagania funkcjonalne oraz strukturalne określone w specyfikacji. Działanie programu powinno umożliwić:
\begin{itemize}
  \item Dodawanie, usuwanie oraz przeszukiwanie elementów w drzewie binarnym;
  \item Wyświetlanie struktury drzewa w różnych formach porządkowania (preorder, inorder, postorder);
  \item Zapis i odczyt drzewa do/z pliku binarnego oraz tekstowego, co pozwoli na ponowne załadowanie drzewa bez utraty danych;
  \item Wczytanie danych z pliku tekstowego i ich przekształcenie w strukturę drzewa BST, zarówno do pustego drzewa, jak i z możliwością łączenia z już istniejącą strukturą.
\end{itemize}

\subsection{Cele techniczne}

Projekt został zaplanowany tak, aby spełnić poniższe cele techniczne:
\begin{itemize}
  \item Implementacja klasy \texttt{BinarySearchTree} realizującej główne operacje na drzewie BST: dodawanie, usuwanie, przeszukiwanie, wyświetlanie oraz zapis i odczyt z pliku;
  \item Stworzenie dodatkowej klasy \texttt{App} odpowiedzialnej za zarządzanie drzewem poprzez menu interaktywne, obsługujące wybory użytkownika oraz wywołujące odpowiednie operacje na drzewie;
  \item Rozdzielenie kodu źródłowego na moduły zgodnie z zasadami programowania obiektowego, w tym podział na pliki nagłówkowe i implementacyjne, aby ułatwić dalszy rozwój i testowanie kodu;
  \item Wykorzystanie narzędzi GitHub do wersjonowania kodu, pracy równoległej oraz rozwiązywania konfliktów;
  \item Stworzenie dokumentacji projektu z wykorzystaniem narzędzia \texttt{Doxygen} oraz dokumentacji w \LaTeX, aby opisać implementację, działanie oraz wyniki końcowe projektu.
\end{itemize}

\subsection{Wymagania dotyczące kontroli wersji i pracy zespołowej}

W projekcie przewiduje się równoległą pracę dwuosobową, co wiąże się z określonymi wymaganiami:
\begin{itemize}
  \item Utworzenie repozytorium GitHub, które będzie śledzić postępy prac nad projektem;
  \item Każdy z członków zespołu będzie odpowiedzialny za stworzenie nowej gałęzi w projekcie, wykonanie co najmniej 5 commitów i późniejsze połączenie tych zmian z główną gałęzią (tzw. \texttt{merge});
  \item Projekt wymaga rozwiązania co najmniej 6 konfliktów w kodzie podczas łączenia gałęzi, co pozwala zdobyć doświadczenie w rozwiązywaniu problemów wynikających z pracy równoległej w zespole;
  \item Zastosowanie narzędzi GitHub do przeglądu kodu, rozwiązywania konfliktów oraz dokumentowania zmian przy pomocy opisowych commitów.
\end{itemize}

\subsection{Oczekiwane wyniki}

Projekt ma na celu uzyskanie aplikacji, która pozwoli użytkownikom na pełną kontrolę nad drzewem BST poprzez interfejs tekstowy. Przyjęte rozwiązania mają umożliwić:
\begin{itemize}
  \item Poprawne zarządzanie drzewem BST, w tym obsługę operacji modyfikujących i przeszukujących strukturę drzewa;
  \item Wykorzystanie plików binarnych i tekstowych do utrwalenia stanu drzewa między sesjami programu;
  \item Wytworzenie dokumentacji projektowej oraz technicznej, która opisze szczegóły implementacyjne oraz przedstawi wykresy procesu pracy grupowej i historii commitów z GitHuba.
\end{itemize}
